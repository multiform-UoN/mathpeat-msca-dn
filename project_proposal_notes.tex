% Project Proposal Notes: Maths and Peat Science
% Author: 
% Date: \today

\documentclass[12pt]{article}
\usepackage[utf8]{inputenc}
\usepackage{amsmath, amssymb}
\usepackage{geometry}
\geometry{a4paper, margin=1in}
\usepackage{url}

\title{EU MCSA Doctoral Training Network}
\author{}
\date{\today}

\begin{document}

\maketitle

\section*{Call Details and Project Summary}

This proposal is for the 2025 call of the Marie Skłodowska-Curie Actions (MSCA) Doctoral Networks (DN): \textbf{HORIZON-MSCA-2025-DN-01}. The deadline for submission is \textbf{25 November 2025, 17:00 Brussels time}.

\subsection*{Project Title and Acronym}
\textbf{MATHPEAT – Mathematical Modelling for Peatland Science and Restoration}

\subsection*{Summary and Objectives}
Peatlands are critical carbon sinks and hydrological buffers with vast implications for climate change, land use, and ecosystem restoration. This project develops rigorous multiscale mathematical and numerical models to capture key physical, chemical, and biological processes in peatlands. It aims to inform restoration strategies, support climate policy, and contribute tools applicable to other porous media systems.

\subsection*{Doctoral Network Structure}
The network will support \textbf{12 to 14 PhD projects}, with approximately 2 PhDs assigned to each work package (WP). Each PhD must include at least \textbf{3 months} of secondment at a different institution; we aim to plan \textbf{6 months total secondments}, either as one long or two 3-month stays. Secondments will strengthen WP integration and collaboration and will take place within the consortium (beneficiaries or associate partners).

Each WP is led by a beneficiary institution. The WP lead hosts one PhD within their WP and supervises another in a different WP, ensuring strong thematic coherence and project-wide leadership. The University of Nottingham will lead the overall project.

Pending confirmations:
\begin{itemize}
    \item WP5 (Continuum Mechanics) tentatively assigned to Giulio Sciarra (Nantes)
    \item WP6 (Numerical Analysis) still require confirmed leads; discussions are ongoing with potential French, German, and Dutch institutions.
\end{itemize}

\section{Work Packages}


\subsection{WP1: Peat Science – Field Work, Ecology, and Data Collection (UHI, Lead: Roxane Andersen)}

Focus: establishing baseline datasets, monitoring restoration projects, GHG emissions, vegetation dynamics.

\subsubsection*{ESR project: Using Cone Penetration Testing (CPT) for sustainable peatland assessment: A low-impact approach to address environmental challenges in geotechnical site investigation}
\paragraph{Supervisors:} Ehsan Jorat
\paragraph{Institution:} Abertay University, University of Nottingham, University of the Highlands and Islands
\paragraph{Other supervisors/collaborators:} David Large – University of Nottingham - david.large@nottingham.ac.uk,   Roxane Andersen – University of the Highlands and Islands - roxane.andersen@uhi.ac.uk 
\paragraph{Keywords:} 
\paragraph{Potential collaborations:} 
\paragraph{Secondments:} 
\paragraph{Objectives:} 
\paragraph{Abstract:} Peatlands are globally recognised as one of the Earth’s most efficient carbon stores, despite covering only around 3\% of the terrestrial surface. Recent studies underscore their critical role in moderating global climate systems and sequestering atmospheric carbon, which is now under severe threat from anthropogenic development, drainage, and climate-induced shifts (Leifeld et al., 2023).

The degradation of peatlands, especially through mechanical disturbance from infrastructure development, drainage for agriculture, and urban expansion, is directly linked to substantial greenhouse gas emissions and loss of ecosystem services (Anderson et al., 2023). Infrastructure projects on peat typically require subsurface geotechnical investigation, yet conventional methods such as coring and borehole drilling are inherently invasive and risk accelerating peat decomposition by altering hydrological and structural balances (Evans et al., 2021).

Cone Penetration Testing (CPTu) offers a non-destructive, continuous, and real-time soil profiling technique that minimises environmental disturbance compared to traditional sampling. Recent advances highlight CPTu’s ability to distinguish soil layering and mechanical transitions within peat profiles, although standard interpretation frameworks for fibrous and amorphous peat materials remain underdeveloped (Ren et al., 2023).

Beyond traditional tip resistance and sleeve friction measurements, the Cone Penetration Test (CPT) equipped with a pore water pressure sensor — commonly referred to as CPTu — offers substantial advantages for peatland investigations. Peat soils are highly sensitive to water content and hydrological variations, and understanding pore water pressure dynamics is essential for predicting both geotechnical stability and ecological health (Nieminen et al., 2022).

CPTu allows for the in-situ measurement of excess pore water pressure during penetration, providing insights into the soil’s consolidation state, drainage characteristics and potentially location of gas pockets in peat. For peat soils, this capability is particularly valuable for identifying variations in peat decomposition and density by correlating pressure response with organic matter content (Zhang et al., 2023).

By leveraging CPTu, this project will not only address mechanical characterisation but also connect subsurface geotechnical parameters with key environmental factors, offering a more integrated framework for sustainable peatland management.

With global climate action and sustainable engineering principles converging, this research seeks to evaluate CPT as a low-impact, technically robust alternative for peatland geotechnical characterisation. The project will contribute to closing the knowledge gap around how CPT measurements relate to the unique mechanical and physical behaviours of peat soils, offering critical insights for both environmental protection and safe infrastructure design.

Aim:
To develop a validated framework for applying Cone Penetration Testing (CPTu) as a primary tool for environmentally responsible geotechnical investigation in peatlands.

Objectives:
1.	To quantify the physical and mechanical properties of peat soils using both CPTu and conventional laboratory testing conducted on peat cores.
2.	To develop empirical correlations between CPTu parameters (tip resistance, sleeve friction, pore water pressure) and peat properties like water content, organic content, fibre content, and undrained shear strength.
3.	To develop a protocol for safe and efficient deployment of in-situ CPT in peatlands as an alternative to conventional coring and laboratory resting.


\subsubsection*{ESR project: [Project title]}
\paragraph{Supervisors:} 
\paragraph{Institution:} 
\paragraph{Other supervisors/collaborators:} 
\paragraph{Keywords:} 
\paragraph{Potential collaborations:} 
\paragraph{Secondments:} 
\paragraph{Objectives:} 
\paragraph{Abstract:} 

\subsection{WP2: Peat Science – Pore Structure, Microscale Biogeochemistry, Lab Tests (University of Waterloo, Lead: Fereidoun Rezanezhad)}

Focus: characterisation of microstructure, organic matter decomposition, microbial activity, small-scale flow and reactive transport.

\subsubsection*{ESR project: PhD in Biogeochemical Reactive Transport Modeling in Complex Dual-Porosity Peat Soils }
\paragraph{Supervisors:} Fereidoun Rezanezhad
\paragraph{Institution:} University of Waterloo, Canada
\paragraph{Keywords:} 
\paragraph{Potential collaborations:} 
\paragraph{Secondments:} 
\paragraph{Objectives:} 
\paragraph{Abstract:} PhD student to work on biogeochemical reactive transport modeling to simulate the fate and fluxes of gases, nutrients and contaminants at key interfaces within the peatlands. The focus will be on the coupling of biogeochemical reactions to transport processes in complex dual-porosity peat soils. PhD student will develop the reactive transport model of greenhouse gas emission dynamics in physically and chemically heterogeneous porous media at a variety of scales, from pore to column to field. The model will include the mathematical and numerical simulations of gas transport dynamics, microbial community response, and seasonal effects, as well as thermodynamically-based modules that predict reaction trajectories of multi-functional soil microbial communities under variable physical-chemical conditions. PhD student will have the opportunity to perform laboratory-scale experiments aimed to analyze the reactive transport processes in porous media and determine rates, mechanisms and products of biogeochemical transformations of nutrients and contaminants and provide experimental datasets for model calibration/validation. 

\subsubsection*{ESR project: PhD in Modeling of Key Environmental Drivers in Greenhouse Gas Emissions in Peatlands}
\paragraph{Supervisors:} Fereidoun Rezanezhad
\paragraph{Institution:} University of Waterloo, Canada
\paragraph{Keywords:} 
\paragraph{Potential collaborations:} 
\paragraph{Secondments:} 
\paragraph{Objectives:} 
\paragraph{Abstract:} PhD student will assemble a dataset of peatland physical, hydrological, and biogeochemical properties (including experimental data and field observations) from global peatlands sites. PhD student will use a robust mathematical and numerical simulations and machine learning models using the data to identify key environmental drivers and predict future changes in greenhouse gas emission rates under future climate scenarios. The goal will be to establish how peatlands in different climatic regions are expected to respond to changing anthropogenic disturbances and climate warming to better understand the peatland carbon and greenhouse gas exchange and the resilience of their carbon source/sink function to disturbance.

\subsection{WP3: Hydrology and Transport – Macroscale Heterogeneity (IDAEA-CSIC, Lead: Marco Dentz)}

Focus: lateral fluxes, water table fluctuations, preferential paths, and multi-year modelling.

\subsubsection*{ESR project: [Project title]}
\paragraph{Supervisors:} 
\paragraph{Institution:} 
\paragraph{Other supervisors/collaborators:} 
\paragraph{Keywords:} 
\paragraph{Potential collaborations:} 
\paragraph{Secondments:} 
\paragraph{Objectives:} 
\paragraph{Abstract:} 

\subsubsection*{ESR project: [Project title]}
\paragraph{Supervisors:} 
\paragraph{Institution:} 
\paragraph{Other supervisors/collaborators:} 
\paragraph{Keywords:} 
\paragraph{Potential collaborations:} 
\paragraph{Secondments:} 
\paragraph{Objectives:} 
\paragraph{Abstract:} 

\subsection{WP4: Continuum Mechanics (Lead: Giulio Sciarra, Nantes)}

Focus: development of poro- and visco-elastic models, large-deformation formulations, interfaces and material heterogeneities.

\subsubsection*{ESR project: [Project title]}
\paragraph{Supervisors:} 
\paragraph{Institution:} 
\paragraph{Other supervisors/collaborators:} 
\paragraph{Keywords:} 
\paragraph{Potential collaborations:} 
\paragraph{Secondments:} 
\paragraph{Objectives:} 
\paragraph{Abstract:} 

\subsubsection*{ESR project: [Project title]}
\paragraph{Supervisors:} 
\paragraph{Institution:} 
\paragraph{Other supervisors/collaborators:} 
\paragraph{Keywords:} 
\paragraph{Potential collaborations:} 
\paragraph{Secondments:} 
\paragraph{Objectives:} 
\paragraph{Abstract:} 

\subsection{WP5: Numerical Analysis (Lead: to be confirmed – candidates in Essen/Eindhoven/France)}

Focus: discretisation methods (FEM/FVM), efficient solvers, time-integration schemes, and uncertainty quantification.

\subsubsection*{ESR project: [Project title]}
\paragraph{Supervisors:} 
\paragraph{Institution:} 
\paragraph{Other supervisors/collaborators:} 
\paragraph{Keywords:} 
\paragraph{Potential collaborations:} 
\paragraph{Secondments:} 
\paragraph{Objectives:} 
\paragraph{Abstract:} 

\subsubsection*{ESR project: [Project title]}
\paragraph{Supervisors:} 
\paragraph{Institution:} 
\paragraph{Other supervisors/collaborators:} 
\paragraph{Keywords:} 
\paragraph{Potential collaborations:} 
\paragraph{Secondments:} 
\paragraph{Objectives:} 
\paragraph{Abstract:} 

\subsection{WP6: Mathematical Modelling and Integration (Padova, Lead: Matteo Camporese)}

Focus: formulation of multiscale models, homogenisation, coupling of transport, mechanics and reactivity, integration with experimental data.

\subsubsection*{ESR project: Bridging Complexity and Practicality in Peatland Hydrology: A Multi-Model Approach}
\paragraph{Supervisors:} Matteo Camporese
\paragraph{Institution:} University of Padova
\paragraph{Other supervisors/collaborators:} Mario Putti, mario.putti@unipd.it, University of Padova, Italy; Lorenzo Sanavia, lorenzo.sanavia@unipd.it, University of Padova, Italy
\paragraph{Keywords:} 
\paragraph{Potential collaborations:} 
\paragraph{Secondments:} 
\paragraph{Objectives:} 
\paragraph{Abstract:} Peat is a complex material that presents unique challenges in hydrological modeling due to its nonlinear dynamics and time-variable properties. This PhD project aims to advance our understanding of peat hydrology through a comprehensive poromechanical modeling approach.
Key objectives:
1.	Use a comprehensive poromechanical model to capture peat's intricate behavior, including swelling and shrinkage cycles.
2.	Quantify temporal variations in critical hydraulic properties, including porosity, specific storage, and hydraulic conductivity, under diverse forcing conditions (e.g., climate variability, vegetation interactions).
3.	Derive simplified characteristic equations for peat hydraulic properties that can be integrated into Richards equation-based solvers.
4.	Establish criteria for determining when these simplified equations are sufficient and when the full poromechanical approach is necessary.
This research aims to bridge the gap between complex peat dynamics and practical hydrological modeling tools, potentially revolutionizing our approach to simulating how water flows within peatlands. The outcomes will have significant implications for understanding and managing these crucial ecosystems in the context of global environmental change.

\subsubsection*{ESR project: [Project title]}
\paragraph{Supervisors:} 
\paragraph{Institution:} 
\paragraph{Other supervisors/collaborators:} 
\paragraph{Keywords:} 
\paragraph{Potential collaborations:} 
\paragraph{Secondments:} 
\paragraph{Objectives:} 
\paragraph{Abstract:} 

\subsection{WP7: Project Management and Coordination (Nottingham, Lead: Matteo Icardi)}

Focus: administration, communication, data management, and cross-WP integration. (No ESR projects directly allocated)

\subsection*{Unassigned Projects (submitted previously, awaiting WP mapping)}

\subsubsection*{ESR project: Multiphase modelling for peat applications}
\paragraph{Supervisors:} John King
\paragraph{Institution:} University of Nottingham
\paragraph{Other supervisors/collaborators:} Matteo Icardi
\paragraph{Keywords:} 
\paragraph{Potential collaborations:} 
\paragraph{Secondments:} 
\paragraph{Objectives:} 
\paragraph{Abstract:} Continuum modelling approaches will be developed.

\subsubsection*{ESR project: Modelling the landscape evolution of peatland}
\paragraph{Supervisors:} David Large
\paragraph{Institution:} University of Nottingham
\paragraph{Other supervisors/collaborators:} To be discussed
\paragraph{Keywords:} 
\paragraph{Potential collaborations:} 
\paragraph{Secondments:} 
\paragraph{Objectives:} 
\paragraph{Abstract:} What are the limits of a given landscape to hold peat? Over what spatial and temporal scales should this be measured?  What peatland landscapes are currently close to these limits? How would they be recognised? 
These questions are some of the biggest unanswered questions in peat science. They sit at the core of strategic restoration, a process that should recognise where to seek carbon gains or minimise losses.  

Current understanding is blinded by the short timeframes of observation and the confounding effects of landscape management.  Therefore to understand the limits imposed by landscape evolution on peatland carbon storage requires models.  

Landscape evolution models are quantitative tools that simulate the evolution of land surfaces over time, driven by factors like erosion, deposition, and isostasy.  They are currently applied to granular soils (sand) for which the processes of transport and erosion are easily modelled.  The challenge is to develop these models for peat and we are now in a strong position to achieve this.  The challenges of LEMs in general are

1. Uncertain parameterisation, especially when dealing with complex processes and material properties. 
2. Balancing the need for realistic representation of processes with computational feasibility

Peat being a non-granular cohesive materials that accumulates on landsurfaces that would normally only erode adds interesting complexity.  Visco-poroelastic mechanic models develop via the MATHPEAT network provide a physical foundation and recent advances in understanding evolution of gullies Mars by headward erosion provide an LSM approach that may be adaptable to peatland where similar processes appear dominant.

The project will combine uncertainty quantification and modelling of the physics to formulate a new generation of landscape evolution models applicable to peat.  



\subsubsection*{ESR project: Microbial ecology of methylotrophic methane production in northern peatland sediments}
\paragraph{Supervisors:} Ozge Eyice
\paragraph{Institution:} Birmingham Biosciences
\paragraph{Other supervisors/collaborators:} Nick Kettridge  or any other suitable supervisor working on modelling
\paragraph{Keywords:} 
\paragraph{Potential collaborations:} 
\paragraph{Secondments:} 
\paragraph{Objectives:} 
\paragraph{Abstract:} Peatlands, though covering only 3\% of the Earth's land surface, are critical to global carbon cycling, acting as long-term carbon sinks by storing approximately 600 Gt of carbon (Joosten, 2009; Yu et al., 2010). These ecosystems, however, are also significant sources of methane (CH₄), contributing 5–10\% of the annual global methane emissions (Frolking et al., 2011). Microbial processes and water table depth tightly control the balance between carbon storage and methane emission from the peatlands. 

Methane emissions can follow three microbial pathways. One of these pathways (methylotrophic methanogenesis) involves the degradation of one-carbon compounds such as methanol, trimethylamine and dimethylsulfide. However, their specific contribution to global methane emissions in peatlands and underlying microbial diversity and metabolism remain understudied. This project will fill this knowledge gap by using field sampling, process measurements, 'omics based analysis and process modelling.

\subsubsection*{ESR project: Suffocating peat: response of nutrient supplies and sphagnum growth to climate extremes in ombrotrophic peatlands}
\paragraph{Supervisors:} Sami Ullah
\paragraph{Institution:} University of Birmingham
\paragraph{Other supervisors/collaborators:} Ozge Eyice
\paragraph{Keywords:} 
\paragraph{Potential collaborations:} 
\paragraph{Secondments:} 
\paragraph{Objectives:} 
\paragraph{Abstract:} Climate extremes including heat waves and soil droughts are becoming more prevalent both in frequency and intensity, which can affect peatlands ecology and their potential for carbon capture. Climate extremes in general reduces soil nutrient supplies thus reducing the ability of peatlands for carbon capture. In contrast, atmospheric CO2 concentration is increasing that leads to the CO2 fertilization effect in peatlands via enhanced photosynthesis and carbon capture. Therefore, understanding the responses of peatlands to climate extreme events under CO2 enriched atmospheres is critical for knowing the resilience of peatlands and their key functions. In the project, we propose to assess how climate extreme events (soil drought and heat events) affects nutrient supplies and carbon capture and whether elevated CO2 compensate and/or ameliorate the negative impacts of climate extremes on nutrient supplies and carbon capture overall. 

The experimental work will involved peatland monoliths (20 cm by 50 cm) subjected to elevated CO2 during the growing season where climate extreme events will be introduced mimicking future climates. Soil water dynamics, nutrient transformations and sphagnum growth will be measured for different moss species and in mixes to understand the resilience of mosses and hence carbon capture under future climates. The data will inform models when predicting the response of peatlands to future climates.

\subsubsection*{ESR project: Analyzing Pattern Formation in Peatlands on Permafrost: A Dynamical Systems and Remote Sensing Approach}
\paragraph{Supervisors:} Ivan Sudakow
\paragraph{Institution:} The Open University
\paragraph{Other supervisors/collaborators:} TBA
\paragraph{Keywords:} 
\paragraph{Potential collaborations:} 
\paragraph{Secondments:} 
\paragraph{Objectives:} 
\paragraph{Abstract:} Permafrost peatlands, occupying significant areas in the Northern Hemisphere, play a critical role in global carbon cycling and climate feedback. Understanding their spatial patterns and dynamic behavior under climate change is vital. This study integrates dynamical systems theory and remote sensing techniques to investigate the complex spatial patterns observed in peatlands on permafrost. Employing nonlinear and stochastic models, we explore phase transitions, bifurcations, and critical phenomena at the permafrost-atmosphere interface, capturing the onset of critical thresholds in methane emissions. Satellite-based remote sensing data supports our model validation, revealing scaling properties and geometrical dynamics of tundra polygons and thermokarst lake formations. Results from this research aim to enhance predictive capabilities concerning permafrost stability, greenhouse gas emissions, and the identification of tipping points within the climate system.

\subsubsection*{ESR project: Peat landslides: Peatslides}
\paragraph{Supervisors:} Gabriele Albertini
\paragraph{Institution:} University of Nottingham
\paragraph{Other supervisors/collaborators:} Matteo Icardi; Luis Espath
\paragraph{Keywords:} 
\paragraph{Potential collaborations:} 
\paragraph{Secondments:} 
\paragraph{Objectives:} 
\paragraph{Abstract:} The Challenges: mechanical failure of peat leads to large losses of peat mass through erosion. The triggers for mechanical failure are changes in boundary conditions:
(1) changes in applied loads due to manmade structures, such as access roads build onto the peat (e.g. for wind farms)
(2) changes in climate (precipitation, evapotranspiration, ecology)
(3) changes in hydrology boundary condition due to manmade remediation attempts (drainage and reversal of past drainages)
(4) biological growth and erosion process where peat intersects with waterways
 
This project builds on current advances in modelling damage of soft, porous solids (hydrogels, biological tissue, glaciers etc) and apply it to peat by integrating hydrology, evapotranspiration, and biological growth. The multi-scale nature of the dynamics is particularly challenging: there are creep like processes occurring over long timescales and dynamic instabilities such as peat-slides that occur within seconds. Additionally, changes in boundary condition now lead to delayed failure which makes prediction challenging.  
 
This project aims to develop new numerical tools for a predictive framework to address the 4 challenges above.

\section{Associate Partners (non beneficiaries)}
\begin{itemize}
    \item Glasgow (Penta)
    \item University of Birmingham (Ozge Eyice, Sami Ullah)
    \item Abertay (Ehsan Jorat)
    \item Pisa (Heltai, Deal.II) or Politecnico di Milano
    \item Galway (Giuseppe Zurlo)
    \item CNR and University of Bari?
    \item Imperial College London (microCT, Martin Blunt)
\end{itemize}

\section{Summer Schools}
Training is a key component of the MATHPEAT Doctoral Network. All ESRs (PhD students) will participate in two interdisciplinary summer schools, combining scientific training with career development and transferable skills.

\begin{itemize}
    \item \textbf{Summer School in Thurso (Scotland)} – focused on peat ecology, field data collection, instrumentation, restoration techniques, and ecological management. Includes field trips and experimental demonstrations.
    \item \textbf{Summer School in Padova (Italy)} – focused on mathematical modelling, multiscale analysis, numerical methods, and software tools relevant to peatland simulation.
\end{itemize}

\section*{TODO – ACTION POINTS}
\begin{enumerate}
    \item Clarify whether field work, experimental equipment, and data collection expenses are eligible under the MSCA-DN budget. \textbf{DONE} – we have up to 1,600 EUR/month/ESR available to cover research training and field work.
    \item Confirm WP5 (Continuum Mechanics) and WP6 (Numerical Analysis) leads.
    \item Check if reuse of acronym "MATHPEAT" is acceptable, or propose alternatives.
    \item Define clear training plan and cross-WP connections for each ESR project.
    \item \textbf{ALL} Start drafting and collecting ESR project descriptions, including objectives, methodologies, and expected outcomes.
    \item Double check Canada’s eligibility – currently appears that Canada cannot be a beneficiary under MSCA-DN (only associated partner, no EU funding). @Fereidoun to confirm with institution.
    \item Keep in mind the 40\% rule – no more than 40\% of total EU contribution can go to beneficiaries in one country. Multiple partners per country are acceptable if this condition is respected and synergies are justified.
\end{enumerate}


\end{document}
