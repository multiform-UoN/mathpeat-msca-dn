% Project Proposal Notes: Maths and Peat Science
% Author: 
% Date: \today

\documentclass[12pt]{article}
\usepackage[utf8]{inputenc}
\usepackage{amsmath, amssymb}
\usepackage{geometry}
\geometry{a4paper, margin=1in}
\usepackage{url}

\title{EU MCSA Doctoral Training Network}
\author{}
\date{\today}

\begin{document}

\maketitle

\section*{Call Details and Project Summary}

This proposal is for the 2025 call of the Marie Skłodowska-Curie Actions (MSCA) Doctoral Networks (DN): \textbf{HORIZON-MSCA-2025-DN-01}. Full details of the call can be found at: \url{https://ec.europa.eu/info/funding-tenders/opportunities/docs/2021-2027/horizon/wp-call/2024/wp4-msca-actions_horizon-2024_en.pdf}. The deadline for submission is \textbf{25 November 2025, 17:00 Brussels time}.

\subsection*{Project Title and Acronym}
\textbf{MATHPEAT – Mathematical Modelling for Peatland Science and Restoration}

\subsection*{Summary and Objectives}
Peatlands are critical carbon sinks and hydrological buffers with vast implications for climate change, land use, and ecosystem restoration. This project develops rigorous multiscale mathematical and numerical models to capture key physical, chemical, and biological processes in peatlands. It aims to inform restoration strategies, support climate policy, and contribute tools applicable to other porous media systems.

\subsection*{Doctoral Network Structure}
The network will support \textbf{12 to 14 PhD projects}, with approximately 2 PhDs assigned to each work package (WP). Each PhD must include at least \textbf{3 months} of secondment at a different institution; we aim to plan \textbf{6 months total secondments}, either as one long or two 3-month stays. Secondments will strengthen WP integration and collaboration and will take place within the consortium (beneficiaries or associate partners).

Each WP is led by a beneficiary institution. The WP lead hosts one PhD within their WP and supervises another in a different WP, ensuring strong thematic coherence and project-wide leadership. The University of Nottingham will lead the overall project.

Pending confirmations:
\begin{itemize}
    \item WP5 (Continuum Mechanics) tentatively assigned to Giulio Sciarra (Nantes)
    \item WP6 (Numerical Analysis) still require confirmed leads; discussions are ongoing with potential French, German, and Dutch institutions.
\end{itemize}


\section{Work Packages}

\begin{enumerate}
    \item Peat Science: field work, ecology, and data collection – establishing baseline datasets, monitoring restoration projects, GHG emissions, vegetation dynamics. (UHI)
    \item Peat Science: pore structure, microscale biogeochemistry, lab tests – characterisation of microstructure, organic matter decomposition, microbial activity, small-scale flow and reactive transport. (Waterloo)
    \item Peat Science: hydrology and transport, macroscale heterogeneity – including lateral fluxes, water table fluctuations, preferential paths, and multi-year modelling. (CSIC IDAEA, Spain)
    \item Continuum Mechanics – development of poro- and visco-elastic models, large-deformation formulations, interfaces and material heterogeneities. (Giulio Sciarra, Nantes?) or Dormieux (Paris?)
    \item Numerical Analysis – discretisation methods (FEM/FVM), efficient solvers, time-integration schemes, and uncertainty quantification. (France/Germany/Netherlands? Essen, W. Boon? Eindhoven?)
        \item Mathematical Modelling and integration – formulation of multiscale models, homogenisation, coupling of transport, mechanics and reactivity, integration with experimental data. (Padova)

    \item Project Management/coordination – administration, communication, data management, and cross-WP integration. (Nottingham)
\end{enumerate}

\section{Associate Partners (non beneficiaries)}
\begin{itemize}
    \item Glasgow (Penta)
    \item University of Birmingham (Ozge Eyice, Sami Ullah)
    \item Abertay (Ehsan Jorat)
    \item Pisa (Heltai, Deal.II) or Politecnico di Milano
    \item Galway (Giuseppe Zurlo)
    \item CNR and University of Bari?
    \item Imperial College London (microCT, Martin Blunt)
\end{itemize}

\section{Summer Schools}
Training is a key component of the MATHPEAT Doctoral Network. All ESRs (PhD students) will participate in two interdisciplinary summer schools, combining scientific training with career development and transferable skills.

\begin{itemize}
    \item \textbf{Summer School in Thurso (Scotland)} – focused on peat ecology, field data collection, instrumentation, restoration techniques, and ecological management. Includes field trips and experimental demonstrations.
    \item \textbf{Summer School in Padova (Italy)} – focused on mathematical modelling, multiscale analysis, numerical methods, and software tools relevant to peatland simulation.
\end{itemize}



\end{document}
\section*{TODO – ACTION POINTS}
\begin{enumerate}
    \item Clarify whether field work, experimental equipment, and data collection expenses are eligible under the MSCA-DN budget. What are the limitations or unit-cost assumptions? DONE, we have up to 1,600 EUR/month/ESR available to cover research training and field work.
    \item Confirm WP5 (Continuum Mechanics) and WP6 (Numerical Analysis) leads.
    \item Check if reuse of acronym "MATHPEAT" is acceptable, or propose alternatives.
    \item Define clear training plan and cross-WP connections for each ESR project.
    \item Start drafting and collecting existing ESR project descriptions, including objectives, methodologies, and expected outcomes.
\end{enumerate}